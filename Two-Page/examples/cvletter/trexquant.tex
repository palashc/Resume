Following are the answers to the questionare: \\ 
\textbf{Ideal job scenario} \\
% What interviewer is looking for: 
% 1. what u're passionate about.
% 2. Ideal Ans: show something related to job profile that you truely like doing. 
% About HFT: highly sophisticated algorithms, co-location, very short-term investment horizons. Ability of HFT is to simultaneously process large volume of information that humans can't do quickly. Competition is in who can do this task faster.
% Strategies
% Market Making: HFT firm characterizes their business as market making. 
% Linear regression, regime detection, Bayesian analysis, 
% My ideal job would involve something challenging. I like challenges. I don't know actually about my ideal job. Or maybe something related to how human mind works. How we think. Why we do what we do. This kind of stuffs. My ideal job would involve a staff that all are passionate about their works. My ideal job would involve designing algorithms etc etc. I am an avid and persevering student. 
My ideal job would involve building machine learning models for some given data. I like to find that which model would work best for given type of data. I liked the field of machine learning, where you can apply mathematical ideas to practical problems. I feel really exhilerated when I see that how such complex real world problems like recommendation, vision are being mathematically modeled. During my previous internships at Pariksha.co and at Adobe, I was given the task (recommendor system and smart geofences), some details about the data and were asked to build a machine learning model. 

% for example, at Adobe, our task was to predict the interest of user in a given product at particular geographical location. Only available data was sparse user-item browse (and purchase) matrix tagged with location. The task was challenging in the sense that forget location, most of the users have not browsed more than five items. We experimented with the data to finally model interest as a function of a) user's intrinsic affinity towards product, b) user affinity based on user-user similarity, c) user's affinity based on location semantics. This breakup was a result of a lot of hypothesis making and testing that through data. This model turned out to work great and was patented internally at Adobe. The project was also presented in Adobe Tech Summit 2017, San Jose. \\  
% Similarily, during my second year internship at Pariksha.co, I was given the task of prediction of intelligence level of users and difficulty level of questions adaptively. Based on the predicted level, recommend questions to the users. We developed and deployed scalable system using Golang and MongoDB database. The platform is currently used by more than 20k students. 


\textbf{Most challenging problem I solved} \\
% I would like to answer both of these questions together. 
At Adobe, we were working on a self chosen and defended (in front of a group of experts) problem. The idea was to create geofences that were smart enough to automatically identify user's interest in given item at given location (Basically asking the question- "What, when and where user needs which thing ?"). Extracting so much information from the sparse data (user-item browse matrix tagged with browsing location) was extremely challenging. At preprocessing time, we made a lot of hypothesis (for example: whether user browse more at his workplace, or at residential locations. What type of products he views? What are the effects of location on a given type of product etc.). These hypothesis were tested that they actually made sense in data or not. 
Based on these refined hypothesises, we decided to model user's interest as a function of a) user's intrinsic affinity towards product, b) user affinity based on user-user similarity, c) user's affinity based on location semantics. 
This model turned out to work great and was patented internally at Adobe. The project was also presented in Adobe Tech Summit 2017, San Jose. \\  
% for example, at Adobe, our task was to predict the interest of user in a given product at particular geographical location. Only available data was sparse user-item browse (and purchase) matrix tagged with location. The task was challenging in the sense that forget location, most of the users have not browsed more than five items. We experimented with the data to finally model interest as a function of a) user's intrinsic affinity towards product, b) user affinity based on user-user similarity, c) user's affinity based on location semantics. This breakup was a result of a lot of hypothesis making and testing that through data. This model turned out to work great and was patented internally at Adobe. The project was also presented in Adobe Tech Summit 2017, San Jose. \\  
% Most challenging problem that we were faced with was during our 3 months internship at Adobe. 
\begin{itemize}
	\item Choose problem yourself and defend it in front of an expert panel.
	\item Work on the problem with sparse data but huge in size
	\item thinking in terms of an abstract concept to implementation details. 
	\item Deployement
	\item Acceptance, patent and awards
\end{itemize}

\textbf{Most memorable accomplishment} \\
The model turned out to work great and was patented internally at Adobe as "Smart Geofencing using location based product affinity". The project was also presented at Adobe Tech Summit 2017, San Jose. Achieving a patent during my undergraduate internship and getting so much acknowledgement was really an accomplishment worth remembering.

\textbf{Smartest person I know personally}\\
The smartest person I know around myself is Professor Mainak Chaudhari. The reason for why he is smart is that he is too dedicated towards his work. He never fails to explain a raised query. He did all his work himself, always puctual and passionate about his work.

\textbf{Top three technical skills at which I excel relative to my peers}  
\begin{itemize}
	\item Implementation of existing code. 
	\item I can implement fast.
	\item I write code that is organized, I can think of problems like deployement issues. 
	\item I have experience from networking side.
	\item Varied profile. Can do a large variety of work.  
\end{itemize}
* Please describe your ideal job.



* Please describe your most memorable accomplishment.
During my research intern at Adobe Research Labs, India, we were given the task of finding and solving an interesting problem in the field of “Mobile Location Intelligence”. We chose to build smart geo-fences that can automatically learn the interest of users and change it’s shape dynamically to target more interested customers. The problem was the data. In the name of data, we only had 

* What is the most challenging problem you solved?
Adobe Patent
Adobe patent

* Who is the smartest person you know personally?  Why?
Mainak, Deepak, Payal, Me :P 
Among all my colleagues whom I personally know, I find myself smarter than them. I learn quickly, I know how to get my work done. I solve problems quickly than others. There was an incident that 


* List the top three technical skills at which you excel relative to your peers.
\begin{itemize}
	\item Fast implementation of existing research and building up on them. Fast understanding of existing huge code base and implementing features on them. 
	\item 
	\item 
\end{itemize}


Pointers about Trexquant
Trexquant is a quantitative investment firm focused on developing and trading statistically-based medium-frequency equity strategies


Pointers about Recruiters: 

Pointers about Job profile: 
One of the most important and popular roles in quantitative finance is Alpha research, which involves developing profitable trading signals based on real-world data.
Unfortunately, quantitative trading is a tight-knit industry and it is difficult to find detailed information about building a successful career in this field.  To address this problem, Trexquant created the Global Alpha Research Program to provide a platform for career growth and advancement and give participants direct experience in buy-side Alpha research.
JD: 
to conduct exciting Alpha research.
Ideal candidate is analytical, creative and persistent at finding strong alphas
Responsibilities: 
Develop market-neutral, medium-frequency Alphas that predict future stock returns. -- 
Investigate and implement recent academic research -- Might show understanding and building up on existing projects fastly. 
Develop algorithms to filter and combine Alphas - Display Algorithmic skills
Parse data sets to be used for future alpha development - Maybe big data skills
Apply machine learning techniques to alpha discovery and portfolio construction - Show ur machine learning skills
