\cvsection{Projects}
\begin{cventries}
	
	%---------------------------------------------------------
	\cventry
	{Undergraduate project under Dr. Harish Karnick} % Job title
	{Multiple Kernel Learning} % Organization
	{IIT Kanpur, India} % Location
	{Jan. - Apr. 2015} % Date(s)
	{
		\begin{cvitems} % Description(s) of tasks/responsibilities
			 \item {Explored relative kernel hilbert space, multiple kernel learning algorithm and hierarchical kernel learning. Project was focused around multiple kernel learning to analyze effects of linear combination of different kernels over classifier.}
			 \item {Extracted surf and convolutional deep-net (pre-trained BVLC GoogleNet model) features for Caltech multiclass object classification dataset containing 102 categories. Implemented Simple MKL algorithm and studied effects of linear combination of distinct kernels on svm classifier.}
		\end{cvitems}
	}
	
	%---------------------------------------------------------
	\cventry
	{Research Project under Dr. Purushottam Kar} % Job title
	{Low rank model for neural networks} % Organization
	{IIT Kanpur, India} % Location
	{Aug. - Nov. 2016} % Date(s)
	{
		\begin{cvitems} % Description(s) of tasks/responsibilities
			\item { Primarily motivated by Robust PCA problem with outlier pursuit which seeks to find the best low-dimensional subspace approximation to highdimensional points after eliminating corruptions. Working on applying similar model to weight matrix of fully connected layer.}
			\item {Learnt about Robust PCA with outlier pursuit and tensor train decomposition of weight matrix.}
		\end{cvitems}
	}
	
	
	%---------------------------------------------------------
	\cventry
	{Course Project under Dr. Harish Karnick} % Job title
	{Automatic Abstract Generation for research papers} % Organization
	{IIT Kanpur, India} % Location
	{Aug. - Nov. 2016} % Date(s)
	{
		\begin{cvitems} % Description(s) of tasks/responsibilities
		\end{cvitems}
	}
	
	%---------------------------------------------------------
	\cventry
	{Course Project under Dr. Gaurav Sharma} % Job title
	{Densecap with NMS Convenet} % Organization
	{IIT Kanpur, India} % Location
	{Aug. - Nov. 2016} % Date(s)
	{
		\begin{cvitems} % Description(s) of tasks/responsibilities
			\item Aim was to choose a problem, implement/reproduce approximate results from an existing paper and finally go beyond that work by identifying some weakness and improving on it. We choose the paper densecap by Karpathy et. al.
			\item Inspired from the future works given by the authors i.e. to discard test-time NMS in favor of a trainable spatial suppression layer and the nms-convenet model presented by [8] we enhanced the proposed model. We implemented and trained the nms-convenet using Keras. 
			\item We were able to enhance the mAP of densecap from 5.698 to 5.76. Meanwhile, we learnt about Convolutional networks, Localization networks, Recognition network and Lanuguage model. The code was implemented using Torch, LuaJIT, Keras and python. 
		\end{cvitems}
	}
	
	
	%---------------------------------------------------------
	\cventry
	{Course Project under Dr. Harish Karnick} % Job title
	{Object(Pedestrian/Two-Wheeler/Three-Wheeler) Detection in Survillience Videos} % Organization
	{IIT Kanpur, India} % Location
	{Jan. - Aug. 2016} % Date(s)
	{
		\begin{cvitems} % Description(s) of tasks/responsibilities
			\item {Identify and classify objects into pedestrians, two-wheelers, three-wheelers and four-wheeler in surveillance video.}
			\item {Performed background-foreground separation to identify moving objects. Tested surf and convolutional deep-net features (BVLC GoogleNet).}
			\item {Tried decision tree, random forest and svm (ovr and ovo) classifiers to predict labels.}
		\end{cvitems}
	}
	%---------------------------------------------------------
	
	
	\cventry
	{Course Project under Dr. Subhajit Roy} % Job title
	{Go-Python-x86 Compiler Design} % Organization
	{IIT Kanpur, India} % Location
	{Jan. - Apr. 2016} % Date(s)
	{
		\begin{cvitems} % Description(s) of tasks/responsibilities
			\item Implemented an end-to-end compiler for a subset of Go language and x86 architecture in python using PLY (python Lex-Yacc).
			\item Provided support for multi-dimensional arrays, nested and recursive procedures etc. Used back tracking algorithm and done short circuiting
		\end{cvitems}
	}
	
	%---------------------------------------------------------
	\cventry
	{Course Project under Dr. Mainak Choudhari} % Job title
	{NachOS} % Organization
	{IIT Kanpur, India} % Location
	{Jan. - Apr. 2016} % Date(s)
	{
		\begin{cvitems} % Description(s) of tasks/responsibilities
			\item Extended the standard system call library of NachOS and implemented system calls pertaining to Fork, Exec, Join, Yield, Sleep and Exit.
			\item Implemented process scheduling algorithms: UNIX Scheduling, FIFO, Round Robin, Shortest Job First and Non-pre-emptive job scheduling.
			\item Programmed page replacement algorithms: Random Page Allocation, FIFO, LRU and LRU Clock to evaluate relative performances
		\end{cvitems}
	}
	%---------------------------------------------------------
	\cventry
	{Course Project under Dr. Sandeep Shukla} % Job title
	{Zoobar Secure} % Organization
	{IIT Kanpur, India} % Location
	{Jan. - Apr. 2016} % Date(s)
	{
		\begin{cvitems} % Description(s) of tasks/responsibilities
			\item Extended zoobar web application to learn about various security concepts including buffer overrun attacks, xss, csrf and sql injection.
			\item Done privilege separation and server side sandboxing on OKWS web server. Created program analysis tools based on symbolic execution.
		\end{cvitems}
	}
	
	%---------------------------------------------------------
	\cventry
	{Subtitle Downloader} % Job title
	{Open Source Projects} % Organization
	{IIT Kanpur, India} % Location
	{Jan. - Apr. 2016} % Date(s)
	{
		\begin{cvitems} % Description(s) of tasks/responsibilities
		\end{cvitems}
	}
	
	\cventry
	{IITK FB Forum App} % Job title
	{} % Organization
	{} % Location
	{Jan. - Apr. 2016} % Date(s)
	{
		\begin{cvitems} % Description(s) of tasks/responsibilities
		\end{cvitems}
	}
\end{cventries}