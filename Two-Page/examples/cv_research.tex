%!TEX TS-program = xelatex
%!TEX encoding = UTF-8 Unicode
% Awesome CV LaTeX Template for CV/Resume
%
% This template has been downloaded from:
% https://github.com/posquit0/Awesome-CV
%
% Author:
% Claud D. Park <posquit0.bj@gmail.com>
% http://www.posquit0.com
%
% Template license:
% CC BY-SA 4.0 (https://creativecommons.org/licenses/by-sa/4.0/)
%


%-------------------------------------------------------------------------------
% CONFIGURATIONS
%-------------------------------------------------------------------------------
% A4 paper size by default, use 'letterpaper' for US letter
\documentclass[11pt, a4paper]{awesome-cv}

% Configure page margins with geometry
\geometry{left=1.4cm, top=.8cm, right=1.4cm, bottom=1.8cm, footskip=.5cm}

% Specify the location of the included fonts
\fontdir[fonts/]

% Color for highlights
% Awesome Colors: awesome-emerald, awesome-skyblue, awesome-red, awesome-pink, awesome-orange
%                 awesome-nephritis, awesome-concrete, awesome-darknight
\colorlet{awesome}{awesome-emerald}
% Uncomment if you would like to specify your own color
% \definecolor{awesome}{HTML}{CA63A8}

% Colors for text
% Uncomment if you would like to specify your own color
% \definecolor{darktext}{HTML}{414141}
% \definecolor{text}{HTML}{333333}
% \definecolor{graytext}{HTML}{5D5D5D}
% \definecolor{lighttext}{HTML}{999999}

% Set false if you don't want to highlight section with awesome color
\setbool{acvSectionColorHighlight}{false}

% If you would like to change the social information separator from a pipe (|) to something else
\renewcommand{\acvHeaderSocialSep}{\quad\textbar\quad}

\newcommand{\ExternalLink}{%
	\tikz[x=1.2ex, y=1.2ex, baseline=-0.05ex]{% 
		\begin{scope}[x=1ex, y=1ex]
			\clip (-0.1,-0.1) 
			--++ (-0, 1.2) 
			--++ (0.6, 0) 
			--++ (0, -0.6) 
			--++ (0.6, 0) 
			--++ (0, -1);
			\path[draw, 
			line width = 0.5, 
			rounded corners=0.5] 
			(0,0) rectangle (1,1);
		\end{scope}
		\path[draw, line width = 0.5] (0.5, 0.5) 
		-- (1, 1);
		\path[draw, line width = 0.5] (0.6, 1) 
		-- (1, 1) -- (1, 0.6);
	}
}

%-------------------------------------------------------------------------------
%	PERSONAL INFORMATION
%	Comment any of the lines below if they are not required
%-------------------------------------------------------------------------------
% Available options: circle|rectangle,edge/noedge,left/right
% \photo{./examples/profile.png}
\name{}{Shubham Agrawal}
\position{Member of Technical Staff, Adobe{\enskip\cdotp\enskip} Computer Science Graduate IIT Kanpur}
% \address{X-304, Amrapali Saphire, Sector 45, Noida, PIN: 201301}

\mobile{(+91) 77-5491-6150}
\email{agshubh191@gmail.com}
% \homepage{submagr.github.io}
\github{submagr}
\linkedin{submagr}
% \gitlab{gitlab-id}
% \stackoverflow{SO-id}{SO-name}
% \twitter{@twit}
% \skype{skype-id}
% \reddit{reddit-id}
% \extrainfo{extra informations}

%\quote{``Must be the change that you want to see in the world."}


%-------------------------------------------------------------------------------
\begin{document}

% Print the header with above personal informations
% Give optional argument to change alignment(C: center, L: left, R: right)
\makecvheader

% Print the footer with 3 arguments(<left>, <center>, <right>)
% Leave any of these blank if they are not needed
\makecvfooter
  {\today}
  {Shubham Agrawal~~~·~~~Curriculum Vitae}
  {\thepage}


%-------------------------------------------------------------------------------
%	CV/RESUME CONTENT
%	Each section is imported separately, open each file in turn to modify content
%-------------------------------------------------------------------------------
%-------------------------------------------------------------------------------
%	SECTION TITLE
%-------------------------------------------------------------------------------
\cvsection{Education}


%-------------------------------------------------------------------------------
%	CONTENT
%-------------------------------------------------------------------------------
\begin{cventries}

%---------------------------------------------------------
  \cventry
    {B.S. in Computer Science and Engineering} % Degree
    {POSTECH(Pohang University of Science and Technology)} % Institution
    {Pohang, S.Korea} % Location
    {Mar. 2010 - PRESENT} % Date(s)
    {
      \begin{cvitems} % Description(s) bullet points
        \item {Got a Chun Shin-Il Scholarship which is given to promising students in CSE Dept.}
      \end{cvitems}
    }

%---------------------------------------------------------
\end{cventries}

%-------------------------------------------------------------------------------
%	SECTION TITLE
%-------------------------------------------------------------------------------
\cvsection{Research Interests}

%-------------------------------------------------------------------------------
%	CONTENT
%-------------------------------------------------------------------------------
Recommender Systems, Machine Learning, eCommerce and Tourism, Adaptive and Conversational Systems  \\ \\
\section{Publication}


\cvitem
	{Nov. 2017}
	{Ankur Garg, Sunav Choudhary, Payal Bajaj, Sweta Agrawal, Abhishek Kedia, and Shubham Agrawal. Smart Geo-fencing with Location Sensitive Product Affinity. ACM SIGSPATIAL 2017}
\section{Patent}
\cvitem
	{Feb. 2017}
	{Ankur Garg, Payal Bajaj, Sweta Agrawal, Abhishek Kedia, and Shubham Agrawal. Smart Geo-fencing with Location Sensitive Product Affinity. Patent Application Number - 15/434,886. (Filed)}
\section{Experience}

\cventry
	{Jul. 2017\\--Ongoing}
	{Adobe Systems India Pvt. Ltd.}
	{Member of Technical Staff (Software Engineer)}
	{}
	{}
	{
		\begin{itemize}
			\item {
				Developed POC of a \textbf{unified annotation tool} which automatically detects the annotation user is trying to make. Trained CNN for detecting tool (highlight/underline/strikethrough/polygons) given user drawings and pdf context. Idea got selected to be shipped with the March release of Adobe Acrobat
			}
			\item {
				Engineered an efficient algorithm for the synchronization of sticky comments between PDFNext (HTML based PDF format) and Classic PDF inside Adobe Acrobat. %The challenge was to associate user click on HTML document to the most relevant textual/graphical content.
			}
			\item {
				Fixed more than ten critical security vulnerabilities (including buffer overflow vulnerabilities and javascript parameter tempering) inside Adobe Acrobat. % Severity ranged from the crash of the application on launching malicious pdf to granting full read/write permissions of the user to the attacker.
			}
		\end{itemize}
	}

\cventry
	{May 2016\\--Jul. 2016}
	{Adobe Systems India Pvt. Ltd.}
	{Research Intern}
	{}
	{}
	{
		\begin{itemize}
	        \item {
	        	The aim was to assist marketers in creating \textbf{smart geo-fences}. The project was focused on segmenting users based on their geo-distributions of mobile app activity, identifying points-of-interest and then suggesting geo-fences customized to each user segment.
	        }        
	        \item { 
		        To unsheathe interest from sparse location tagged browsing data, algorithm captures intrinsic interest of user, trends at semantically similar locations and similarity between products and users
		    }
		    \item {
		    	Achieved f1 score (24.89\%) was significantly higher than geofence designed using Matrix Factorization (18.16\%). \hfill (\href{https://drive.google.com/file/d/1uoMtvSSBmur6VNDnTeImWG9TrFckfpQF/view?usp=sharing}{ACM SIGSPATIAL '17 paper \ExternalLink} ,\href{https://drive.google.com/file/d/1nfCv13-7V3BElFPLBfFI\_LF0RRcsdNvy/view?usp=sharing}{ paper presentation \ExternalLink}, \href{https://research.adobe.com/project/smart-geo-fencing/}{project page \ExternalLink},  \href{https://www.youtube.com/watch?v=PS0imh951ZE}{Adobe Tech Summit talk video \ExternalLink})  
		    }
%		    \item {
%		    	The project was showcased in Adobe Tech Summit, 2017 (an annual internal research and engineering conference) 
%		    }
		\end{itemize}
	}
	
\cventry
{May 2015\\-- Jul. 2015}
{Pariksha.co}
{(Startup for digizing education) Research \& Development Intern}
{}
{}
{
	\begin{itemize}
		\item {Engineered an algorithm that adaptively recommends questions depending upon student’s performance and question ratings. The algorithm uses modified version of model-based collaborative filtering}
		\item {Modeled and programmed scalable \textbf{adaptive recommender system} as a microservice using the GO language and MongoDB database }
		\item {Implemented Pariksha Practice Section for adaptive content and a Gamification engine with impact on more than 50K students}
	\end{itemize}
}
\cvsection{Projects}
\begin{cventries}
	
	%---------------------------------------------------------
	\cventry
	{Undergraduate project under Dr. Harish Karnick} % Job title
	{Multiple Kernel Learning} % Organization
	{IIT Kanpur, India} % Location
	{Jan. - Apr. 2015} % Date(s)
	{
		\begin{cvitems} % Description(s) of tasks/responsibilities
			 \item {Explored relative kernel hilbert space, multiple kernel learning algorithm and hierarchical kernel learning. Project was focused around multiple kernel learning to analyze effects of linear combination of different kernels over classifier.}
			 \item {Extracted surf and convolutional deep-net (pre-trained BVLC GoogleNet model) features for Caltech multiclass object classification dataset containing 102 categories. Implemented Simple MKL algorithm and studied effects of linear combination of distinct kernels on svm classifier.}
		\end{cvitems}
	}
	
	%---------------------------------------------------------
	\cventry
	{Research Project under Dr. Purushottam Kar} % Job title
	{Low rank model for neural networks} % Organization
	{IIT Kanpur, India} % Location
	{Aug. - Nov. 2016} % Date(s)
	{
		\begin{cvitems} % Description(s) of tasks/responsibilities
			\item { Primarily motivated by Robust PCA problem with outlier pursuit which seeks to find the best low-dimensional subspace approximation to highdimensional points after eliminating corruptions. Working on applying similar model to weight matrix of fully connected layer.}
			\item {Learnt about Robust PCA with outlier pursuit and tensor train decomposition of weight matrix.}
		\end{cvitems}
	}
	
	
	%---------------------------------------------------------
	\cventry
	{Course Project under Dr. Harish Karnick} % Job title
	{Automatic Abstract Generation for research papers} % Organization
	{IIT Kanpur, India} % Location
	{Aug. - Nov. 2016} % Date(s)
	{
		\begin{cvitems} % Description(s) of tasks/responsibilities
			\item The important sentences are first extracted from the paper text and fed to an abstractive model which outputs the final summary for the paper
			\item Word frequency based scores, text rank and latent semantic analysis were experimented for extraction. We uses a RNN encoder-decoder network to generate the final abstract. Model was evaluated using ROGUE metric.  
		\end{cvitems}
	}
	
	%---------------------------------------------------------
	\cventry
	{Course Project under Dr. Gaurav Sharma} % Job title
	{Densecap with NMS Convenet} % Organization
	{IIT Kanpur, India} % Location
	{Aug. - Nov. 2016} % Date(s)
	{
		\begin{cvitems} % Description(s) of tasks/responsibilities
			\item Aim was to choose a problem, implement/reproduce approximate results from an existing paper and finally go beyond that work by identifying some weakness and improving on it. We choose the paper \textit{DenseCap: Fully Convolutional Localization Networks for Dense Captioning}
			\item Inspired from the future works given by the authors i.e. to discard test-time NMS in favor of a trainable spatial suppression layer and the nms-convenet model presented by work-\textit{A convnet for non-maximum suppression},  we enhanced the proposed model. We implemented and trained the nms-convenet using Keras. 
			\item We were able to enhance the mAP of densecap from 5.698 to 5.76. Meanwhile, we learnt about Convolutional networks, Localization networks, Recognition network and Lanuguage model. The code was implemented using Torch, LuaJIT, Keras and python. 
		\end{cvitems}
	}
	
	
	%---------------------------------------------------------
	\cventry
	{Course Project under Dr. Harish Karnick} % Job title
	{Object(Pedestrian/Two-Wheeler/Three-Wheeler) Detection in Survillience Videos} % Organization
	{IIT Kanpur, India} % Location
	{Jan. - Aug. 2016} % Date(s)
	{
		\begin{cvitems} % Description(s) of tasks/responsibilities
			\item {Identify and classify objects into pedestrians, two-wheelers, three-wheelers and four-wheeler in surveillance video.}
			\item {Performed background-foreground separation to identify moving objects. Tested surf and convolutional deep-net features (BVLC GoogleNet).}
			\item {Tried decision tree, random forest and svm (ovr and ovo) classifiers to predict labels.}
		\end{cvitems}
	}
	%---------------------------------------------------------
	
	
	\cventry
	{Course Project under Dr. Subhajit Roy} % Job title
	{Go-Python-x86 Compiler Design} % Organization
	{IIT Kanpur, India} % Location
	{Jan. - Apr. 2016} % Date(s)
	{
		\begin{cvitems} % Description(s) of tasks/responsibilities
			\item Implemented an end-to-end compiler for a subset of Go language and x86 architecture in python using PLY (python Lex-Yacc).
			\item Provided support for multi-dimensional arrays, nested and recursive procedures etc. Used back tracking algorithm and done short circuiting
		\end{cvitems}
	}
	
	%---------------------------------------------------------
	\cventry
	{Course Project under Dr. Mainak Choudhari} % Job title
	{NachOS} % Organization
	{IIT Kanpur, India} % Location
	{Jan. - Apr. 2016} % Date(s)
	{
		\begin{cvitems} % Description(s) of tasks/responsibilities
			\item Extended the standard system call library of NachOS and implemented system calls pertaining to Fork, Exec, Join, Yield, Sleep and Exit.
			\item Implemented process scheduling algorithms: UNIX Scheduling, FIFO, Round Robin, Shortest Job First and Non-pre-emptive job scheduling.
			\item Programmed page replacement algorithms: Random Page Allocation, FIFO, LRU and LRU Clock to evaluate relative performances
		\end{cvitems}
	}
	%---------------------------------------------------------
	\cventry
	{Course Project under Dr. Sandeep Shukla} % Job title
	{Zoobar Secure} % Organization
	{IIT Kanpur, India} % Location
	{Jan. - Apr. 2016} % Date(s)
	{
		\begin{cvitems} % Description(s) of tasks/responsibilities
			\item Extended zoobar web application to learn about various security concepts including buffer overrun attacks, xss, csrf and sql injection.
			\item Done privilege separation and server side sandboxing on OKWS web server. Created program analysis tools based on symbolic execution.
		\end{cvitems}
	}
	
	%---------------------------------------------------------
	\cventry
	{Subtitle Downloader} % Job title
	{Open Source Projects} % Organization
	{IIT Kanpur, India} % Location
	{Dec. 2013} % Date(s)
	{
		\begin{cvitems} % Description(s) of tasks/responsibilities
			\item Wrote a python script that scrapes and extract the subtitle from subscene.com.
		\end{cvitems}
	}
	
	\cventry
	{IITK FB Forum App} % Job title
	{} % Organization
	{} % Location
	{May 2014} % Date(s)
	{
		\begin{cvitems} % Description(s) of tasks/responsibilities
			\item Created an online discussion website for students and connected it with Facebook to make it more user friendly
			\item Used PHP and PostgreSQL for back-end and hosted application on Heroku. Used Facebook SDK for JavaScript for facebook login and Canvas frame for user interface
		\end{cvitems}
	}
\end{cventries}
%-------------------------------------------------------------------------------
%	SECTION TITLE
%-------------------------------------------------------------------------------
\cvsection{Scholastic Achievements}

%-------------------------------------------------------------------------------
%	CONTENT
%-------------------------------------------------------------------------------
\begin{cvhonors}
	%---------------------------------------------------------
%	\cvhonor
%	{Paper} % Award
%	{Smart Geo-fencing using Location Sensitive Product Affinity (ACM SIGSPATIAL 2017)} % Event
%	{California, USA} % Location
%	{2017} % Date(s)
	
	%---------------------------------------------------------
%	\cvhonor
%	{Patent} % Award
%	{Smart Geo-fencing using Location Sensitive Product Affinity (internally accepted at Adobe)} % Event
%	{India} % Location
%	{2016} % Date(s)
%	
	%---------------------------------------------------------
	\cvhonor
	{All India Rank 191 } % Award
	{IIT-JEE Advanced (among 150,000 candidates)} % Event
	{} % Location
	{2013} % Date(s)
	
	%---------------------------------------------------------
	\cvhonor
	{Academic Excellence Award} % Award
	{(awarded to top 7\% students in the institute) } % Event
	{} % Location
	{2014} % Date(s)
	
	%---------------------------------------------------------
	\cvhonor
	{Best Rookie Team} % Award
	{BAJA Student India, an inter-collegiate all terrain vehicle design competition} % Event
	{} % Location
	{2015} % Date(s)
	
	%---------------------------------------------------------
	\cvhonor
	{All India Rank 1234} % Award
	{JEE Mains (among 1,400,000 candidates)} % Event
	{} % Location
	{2013} % Date(s)
	
%	%---------------------------------------------------------
%	\cvhonor
%	{Online qualifier rank 
%	\href{https://www.codechef.com/rankings/SNCKQL17?order=asc&search=agartudu&sortBy=rank}{1248}
%	} % Award
%	{Pre elimination round A rank \href{https://www.codechef.com/rankings/SNCKPA17?order=asc&search=agartudu&sortBy=rank}{729}, Codechef Smack down (among 21,000 teams)} % Event
%	{India} % Location
%	{2017} % Date(s)
%	
%	%---------------------------------------------------------
%	\cvhonor
%	{Shortlisted for Interview Round} % Award
%	{Kishore Vigyan Protsahan Yojna (KVPY)} % Event
%	{India} % Location
%	{2013} % Date(s)
%	
\end{cvhonors}
\cvsection{Teaching}
\begin{cventries}
	\cventry
	{Counselling Service, IIT Kanpur} % Affiliation/role
	{Academic Mentor, Introduction to Electrodynamics (PHY103)} % Organization/group
	{July 2014 - Apr. 2015} % Location
	{} % Date(s)
	{
		\begin{cvitems} % Description(s) of experience/contributions/knowledge
			\item {Conducted regular tutorial classes at the institute and hall level}
			\item {Guided 2 students out of the Academic Probation Program (AP) by constant academic and emotional support }
		\end{cvitems}
	}
\end{cventries}


% %-------------------------------------------------------------------------------
%	SECTION TITLE
%-------------------------------------------------------------------------------
\cvsection{Skills}


%-------------------------------------------------------------------------------
%	CONTENT
%-------------------------------------------------------------------------------
\begin{cvskills}

%---------------------------------------------------------
  \cvskill
    {Machine Learning} % Category
    {Scikit Learn, Keras, Theano, Lua, Matlab} % Skills

%---------------------------------------------------------
  \cvskill
    {Web} % Category
    {Go, Javascript, Material Design, CSS, HTML, PHP, MYSQL, MongoDB, CakePHP} % Skills
%---------------------------------------------------------
	\cvskill
	{Programming} % Category
	{Python, C/C++} % Skills

%---------------------------------------------------------
  \cvskill
  {Tools} % Category
  {Ubuntu, Windows, Vim, \LaTeX, SolidWorks} % Skills
\end{cvskills}

%-------------------------------------------------------------------------------
%	SECTION TITLE
%-------------------------------------------------------------------------------
\cvsection{Courses}


%-------------------------------------------------------------------------------
%	CONTENT
%-------------------------------------------------------------------------------
\begin{cvskills}

%---------------------------------------------------------
  \cvskill
    {Machine Learning} % Category
    {
    	Bayesian Machine Learning, Natural Language Processing, Recent Advances in Computer Vision, Optimization Techniques, Machine Learning Tools
    } % Skills

%---------------------------------------------------------
  \cvskill
    {Systems} % Category
    {
    	Computer Architecture, Operating Systems, Compiler Design, Computer Networks,
    	Computer Security, Computer Organization, Principles of Database Systems
    } % Skills

%---------------------------------------------------------
  \cvskill
    {Theory} % Category
    {Advanced Algorithms, Data Structures and Algorithms, Theory of Computation, Linear Algebra, Probability and Statistics} % Skills
	
%---------------------------------------------------------

\end{cvskills}

%-------------------------------------------------------------------------------
%	SECTION TITLE
%-------------------------------------------------------------------------------
\cvsection{Extracurricular Activity}


%-------------------------------------------------------------------------------
%	CONTENT
%-------------------------------------------------------------------------------
\begin{cventries}
%---------------------------------------------------------
	\cventry
	{Chassis Head} % Affiliation/role
	{IITK Motorsports, BAJA Student India} % Organization/group
	{Oct. 2013 - Jan. 2015} % Location
	{} % Date(s)
	{
		\begin{cvitems} % Description(s) of experience/contributions/knowledge
			\item {Built the 3rd lightest All-Terrain vehicle of the country in a team of 20 members to compete against 44 national teams}
			\item { Designed the chassis on Solidworks and did FE Analysis and weight optimization on ANSYS WorkBench.}
			\item {
				Contacted dealers for tubes, welding and supervised the whole manufacturing process
			}
			\item { Bagged 4th position in the acceleration event, Best Rookie Team trophy and awarded as Design Finalists in BAJA Student India’15 }
		\end{cvitems}
	}

  
   %---------------------------------------------------------
\end{cventries}



%-------------------------------------------------------------------------------
\end{document}
